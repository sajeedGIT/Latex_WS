\documentclass[legalpaper, 12pt, addpoints]{exam}
\usepackage[margin=1in]{geometry}
\usepackage[utf8]{inputenc}
\usepackage{graphics}
\usepackage{color}
\usepackage{amssymb}
\usepackage{amsmath}
\usepackage{enumitem}
\usepackage{xcolor}
\usepackage{cancel}
\usepackage{ragged2e}
\usepackage{graphicx}
\usepackage{multicol}
\usepackage{color}
\usepackage{tikz}
\usepackage{circuitikz}
\usepackage{siunitx}
\CorrectChoiceEmphasis{\itseries\color{red}}

\begin{document}
\begin{coverpages}

%---uncomment to add a custom header (replace {header-cufm.png})---- %
%\begin{figure}[t]
%\includegraphics[width=1\textwidth,height=1.2\textheight,keepaspectratio]{%header-cufm.png}
%\end{figure}

\begin{center}
Spring 2018 \\
\textbf{Precalculus' Final Examination } \\
nth Semester \\
Section 000
\end{center}
\extraheadheight{-0.8in}

\vspace{0.25in}
\parbox{6in}{
\textbf{Objective:} Assess understanding of function behavior and calculus readiness. This exam also aims to provide a comprehensive assessment on concepts and definitions that are necessary to be successful in further Mathematics courses.)}
\vspace{0.15in}

\parbox{6in}{{\textbf{General Instructions:} Read carefully each exercise. Fill in your \textit{scantron} with a pencil and circle the correct answer on paper as well. Scratch paper is not allowed under any circumstances. All your work must be done in these pages.}
\vspace{0.15in}}

 \begin{center}
\fbox{\fbox{\parbox{6in}{
\begin{itemize}
    \item You have up to 120 minutes.
    
    \item Every item on the test awards 2 points for each correct answer, for a maximum possible score of 100 points.
    
    \item Non-graphing calculators are allowed. TI-84 or similar, including smart devices, are prohibited. 

    
    \item One half-page formula sheet printed in black ink and showing the instructor's authorization may be used. Any other form of aid is not allowed. 
    
    \item Mere suspicion of cheating, sharing calculators or using any unfair means of aid is enough to get your test withdrawn.
        
    \item When you are done, turn in the examination, your \textit{scantron} and your formula sheet. Failure to do so will result in an automatic failing grade.


    
\end{itemize} }}}
\vspace{0.2in}
\end{center}
\vspace{0.15in}
\runningheadrule \extraheadheight{0.14in}

\lhead{\ifcontinuation{Question \ContinuedQuestion\ continues\ldots}{}}
\runningheader{Precalculus}{Final Examination}{Spring 2018}
\runningfooter{4th Semester}
              {\thepage\ of \numpages}
              {Version A}
\vspace{0.15in}

\vspace{0.1in}

%-comment out the next line to display point value for each question -%
\nopointsinmargin
\setlength\linefillthickness{0.1pt}
\setlength\answerlinelength{0.1in}
\end{coverpages}

\parbox{6in}{\textsc{{SIDE A}}

\vspace{0.15in}

\parbox{5in}{
{\textsc{\textbf{Part I.} Network Theory Questions.}}}}

\vspace{0.15in}
\hrule 
\vspace{0.1in}
\begin{questions}

\question The current $I_y$ flowing through $660 \Omega$ resistance  is (Refer Figure~\ref{fig:1}):
\begin{figure}[h!]
\centering
\begin{circuitikz}[american voltages]
\draw
(0,4) to [V, v_= $3V$, i = $I_x$] (0,0);
\draw
 (0,4)     to [short]      (3,4)
      to [R, i= $I_y$, l=$660 \Omega$] (6,4);
\draw
(0,0) to [R, l=$330\Omega$] (3,0);
\draw
(3,4) to [short] (3,3)
      to [R, l=$660\Omega$] (6,3)
      to [short] (6,4);
\draw
(6,3) to [short] (3,0)
      to [R, l_=$330 \Omega$] (0,4);      
\end{circuitikz}
\caption{Q.No.1}
\label{fig:1}
\end{figure}

\begin{oneparchoices}
    \choice $I_x$
    \choice $I_x/2$
    \CorrectChoice $I_x/4$
    \choice $I_x/3$
\end{oneparchoices}

\question The voltage across $660 \Omega$ resistance is (refer Figure~\ref{fig:2}):
\begin{figure}[h!]
\centering
\begin{circuitikz}[american ]
\draw
(0,0) to [short, i=$I_x$] (0,4)
      to [R, l= $330 \Omega$] (4,4)
      to [R, l= $660 \Omega$] (8,4)
      to [short]  (8,0);
\draw      
(0,0) to [V, v= $3V$] (8,0);
\draw
(4,4) to [short] (4,3)
      to [R, l_= $660 \Omega$] (8,3);

\end{circuitikz}
\caption{Q.No.2}
\label{fig:2}
\end{figure}
\begin{oneparchoices}
    \choice $0.65 V$
    \choice $1.5 V$
    \choice $0.72 V$
    \CorrectChoice $0.75 V$
    
\end{oneparchoices}
\question The current $I_x$ and $I_y$ are (refer Figure~\ref{fig:3}).
\begin{figure}[h!]
\centering
\begin{circuitikz}[american]
\draw
(0,0) to [isource, l=$1A$] (4,0)
      to [R, l= $5 \Omega$, i=$I_x$] (8,0)
      to [short] (8,2)
      to [short] (4,2)
      to [isource, l=$5A$] (4,0)
      to [R, l = $1 \Omega$, i=$I_y$] (4, -3)
      to [short] (0, -3)
      to [short] (0,0);
\end{circuitikz}
\caption{Q.No.3}
\label{fig:3}
\end{figure}
\begin{oneparchoices}
    \choice $-1A, 5A$
    \choice $ 5A, 1A$
    \choice $1A, 5A$
    \CorrectChoice $5A, -1A$
    
\end{oneparchoices}

\question The current $I_1$ and $I_2$ of the circuit shown in Figure~\ref{fig:4} are giving by: 
\begin{figure}[h!]
\centering
\begin{circuitikz}[american]
\draw
(0,0) to [isource, l=$10A$] (0,4)
      to [short]  (8,4)
      to [R, l=$2 \Omega$, i=$I_2$] (8,0)
      to [short] (0,0) ;
\draw
(4,4) to [R, l=$3 \Omega$, i=$I_1$] (4,0);
\end{circuitikz}
\caption{Q.No.4}
\label{fig:4}
\end{figure}
\begin{oneparchoices}
    \CorrectChoice $4A, 4A$
    \choice $6A, 6A$
    \choice $4A, 6A$
    \choice $6A, 4A$
\end{oneparchoices}

\question Referring to the circuit of the Figure~\ref{fig:5}, a $35V$ source is connected to a series circuit of $600 \Omega$ and $R$. If a voltmeter of internal resistance $1.2 \SI{}{k \Omega}$ is connected across $600 \Omega$, it reads $5V$. The value of $R$ is
\begin{figure}[h!]
\centering
\begin{circuitikz}[american]
\draw
(0,4) to [battery2, l=$35V$] (0,0)
      to [short] (4,0)
      to [R, l=$R$] (4,2)
      to [R, l=$600\Omega$] (4,4)
      to [short] (0,4);
\draw
(6,4) to [voltmeter] (6, 2)
      to [short] (4,2);
\draw      
(4,4) to [short] (6,4);            
\end{circuitikz}
\caption{Q.No.5}
\label{fig:5}
\end{figure}
\begin{oneparchoices}
    \CorrectChoice $1.2 \SI{}{k\Omega}$
    \choice $2.4 \SI{}{k\Omega}$
    \choice $1.4 \SI{}{k\Omega}$
    \choice $3.4 \SI{}{k\Omega}$
\end{oneparchoices}

\question  The equivalent resistance of the circuit given in Figure~\ref{fig:6} is given by
\begin{figure}[h!]
\centering
\begin{circuitikz}[american]
\draw
(-1,0) to [short, o-] (0,0);
\draw
(0,0) to [R, l=$1\textrm{k}\Omega$] (2,0)
      to [R, l=$1\textrm{k}\Omega$] (6,0)
      to [R, l=$3\textrm{k}\Omega$] (10,0)
      to [R, l=$5\textrm{k}\Omega$] (14,0)
      to [short] (15,0)
      to [short] (15,-4)
      to [short, -o] (-1,-4);
\draw
(3,-1) to [R, l_=$1\textrm{k}\Omega$] (5, -1);
\draw
(3, -1) to [short] (3,0);
\draw
(5, -1) to [short] (5,0);
\draw
(7,-1) to [R, l_=$3\textrm{k}\Omega$] (9, -1);
\draw
(7, -1) to [short] (7,0);
\draw
(9, -1) to [short] (9,0);
\draw
(11,-1) to [R, l_=$5\textrm{k}\Omega$] (13, -1);
\draw
(11, -1) to [short] (11,0);
\draw
(13, -1) to [short] (13,0);

\end{circuitikz}
\caption{Q.No.6}
\label{fig:6}
\end{figure}
\begin{oneparchoices}
    \choice $\SI{4}{k\Omega}$
    \CorrectChoice $\SI{10}{k\Omega}$
    \choice $\SI{5.5}{k\Omega}$
    \choice $\SI{5}{k\Omega}$
\end{oneparchoices}




\vspace{0.10in}

\end{questions}
\end{document}
 