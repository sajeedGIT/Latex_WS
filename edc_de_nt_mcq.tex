\documentclass[a4, 12pt, addpoints]{exam}
\usepackage[margin=0.9in]{geometry}
\usepackage[utf8]{inputenc}
\usepackage{graphics}
\usepackage{color}
\usepackage{amssymb}
\usepackage{amsmath}
\usepackage{enumitem}
\usepackage{xcolor}
\usepackage{cancel}
\usepackage{ragged2e}
\usepackage{graphicx}
\usepackage{multicol}
\usepackage{color}
\usepackage{tikz}
\usepackage{circuitikz}
\usepackage{siunitx}
\usepackage{longtable}
\usepackage{float}
\usepackage{parskip}
\CorrectChoiceEmphasis{\itseries\color{red}}

\runningfooter%
    {}
    {\textbf{D.I.E.T., Satara}}
    { \thepage
    %\iflastpage{}{
%        \textbf{Question \thequestion }%
%        \ifcontinuation{ -- continued}{}\\%
%        \oddeven{\textbf{TURN OVER}}{}
%        }
    }
%MOVE THIS FOOTER TO THE RIGHT BY 0.4in, same for the header, but I've left it out as the header is even longer which lots of if conditions.

\pagestyle{headandfoot}
\begin{document}
\def\arraystretch{2}
\begin{longtable}{lp{0.65\textwidth}p{0.15\textwidth}r}
\multicolumn{4}{c}{\includegraphics[width= \textwidth]{dietlogo}} \\ 
\hline 
\multicolumn{4}{c}{Written Test for Diploma Interview} \\
%\multicolumn{4}{c}{}
\end{longtable}
\parbox{5in}{
\hfill Date: 15/06/2024 \hfill Time:1 hr  \hfill Marks:25 \hfill} \\
Note: All questions carries equal marks

\vspace{0.1in}
\hrule 
\vspace{0.1in}
\begin{questions}
\question Forbidden energy gap for silicon semiconductor is:\\[0.3cm]
\begin{oneparchoices}
\choice 1.2 eV
\choice 0.7 eV
\choice 1.1 eV
\choice 0.4 eV
\end{oneparchoices} 
\question If silicon diode is operating in forward bias in a circuit with 12 V supply and 240 $\Omega$ series resistance, then what is the voltage drop across the diode. \\[0.3cm]
\begin{oneparchoices}
\choice 1.5 V
\choice 0.4 V
\choice 1.1 V
\choice 0.7 V
\end{oneparchoices}  
\question Which of the following is the trivalent doping element?:\\[0.3cm]
\begin{oneparchoices}
\choice Arsenic
\choice Boron
\choice Phosphorous
\choice Antimony
\end{oneparchoices}  
\question The ripple factor for the bridge rectifier is:\\[0.3cm]
\begin{oneparchoices}
\choice 0.406
\choice 1.21
\choice 1.10
\choice 2.22
\end{oneparchoices}  
\question According to barkhausen criteria the loop gain $\beta v$ of the oscillator must be equal to --------.\\[0.3cm]
\begin{oneparchoices}
\choice 0
\choice 1
\choice 0.8
\choice -1
\end{oneparchoices} 
\question The BJT as a switch is is operated in one of the following:
\begin{oneparchoices}
\choice Only saturation region
\choice Active region
\choice Only cut off region
\choice Both saturation and cut off region
\end{oneparchoices}  

\question A DC power supply has no load voltage of 30 V and full load voltage of 25 V at full load current of 1 A. Its output resistance and load regulation respectively are. \\[0.3cm]
\begin{oneparchoices}
\choice $5 \Omega$ and $20 \%$ 
\choice $25 \Omega$ and $20 \%$
\choice $5 \Omega$ and $ 16.7 \%$
\choice $25 \Omega$ and $16.7 \%$
\end{oneparchoices} 
\question When PN junction is forward biased:\\[0.3cm]
 \begin{oneparchoices}
\choice Deletion region decreases
\choice Minority carriers are not affected
\choice Holes and electrons moves away from each other
\choice All of above.
\end{oneparchoices} 
\question According to boolean algebraic theorem  the expression $A (A + B)$ is equivalent to: \\[0.3cm]
 \begin{oneparchoices}
\choice $\pmb{A + B} $ 
\choice $\pmb{B}$
\choice $\pmb{A}$
\choice $\pmb{AB}$
\end{oneparchoices} 
\question What will be the o/p of the given logic gate of Figure~\ref{lg}?
\begin{figure}[H]
\centering
\begin{circuitikz}[american voltages]
\draw
 
 (0,2) node[nor port] (mynor1) {}
(0,0) node[nor port] (mynor2) {}
(2,1) node[nor port] (mynor) {}
(mynor1.out) -- (mynor.in 1)
(mynor2.out) -- (mynor.in 2)
(4,1) node[nor port] (mynor3) {}
(mynor3.west) -- (mynor.out)
(mynor3.in 1) -- (mynor3.in 2)
(mynor1.in 1) -- (mynor1.in 2)
(mynor2.in 1) -- (mynor2.in 2)
(mynor1.west) -- (-2,2)
(mynor2.west) -- (-2,0)
(-2.2,2) node{A}
(-2.2,0) node{B}
(mynor3.east) node {~~~Q} ;  
\end{circuitikz}
\caption{Q.No.10}
\label{lg}
\end{figure}
\begin{oneparchoices}
\choice NOR
\choice NAND
\choice AND
\choice OR
\end{oneparchoices} 
\question The decimal number representation of the following number $ (1~1~0~1~0~1)_2 $ is: \\[0.3cm]
\begin{oneparchoices}
\choice $ (53)_{10} $ 
\choice $ (12)_{10} $
\choice $ (45)_{10}$
\choice $ (67)_{10} $
\end{oneparchoices} 
\question Which among the following is a current controlled device?
\begin{oneparchoices}
\choice MOSFET
\choice BJT
\choice IGBT
\choice JFET
\end{oneparchoices} 
\question The storage delay time can be reduced considerably by preventing transistor from going into saturation. This is achieved by connecting the schottky diode between -------- and --------:\\[0.3cm]
\begin{oneparchoices}
\choice Base and Collector
\choice Base and Emitter
\choice Emitter and Collector
\choice In series with Base. 
\end{oneparchoices}  
\question Gate to Source voltage must be ------- the threshold voltage for enhancement type MOSFET to be cut off.\\[0.3cm]
\begin{oneparchoices}
\choice Greater than
\choice Equal to
\choice Less than
\choice All of the above
\end{oneparchoices}  
\question An equivalent base 2 number of $(13)_{10}$ is:\\[0.3cm]
\begin{oneparchoices}
\choice $(0~1~0~1)_2$ 
\choice $(1~1~0~1)_2$
\choice $(1~1~1~1)_2$
\choice $(1~0~0~1)_2$
\end{oneparchoices} 
\question The current $I_y$ flowing through $660 \Omega$ resistance  is (Refer Figure~\ref{fig:1}):
\begin{figure}[H]
\centering
\begin{circuitikz}[american voltages]
\draw
(0,4) to [V, v_= $3V$, i = $I_x$] (0,0);
\draw
 (0,4)     to [short]      (3,4)
      to [R, i= $I_y$, l=$660 \Omega$] (6,4);
\draw
(0,0) to [R, l=$330\Omega$] (3,0);
\draw
(3,4) to [short] (3,3)
      to [R, l=$660\Omega$] (6,3)
      to [short] (6,4);
\draw
(6,3) to [short] (3,0)
      to [R, l_=$330 \Omega$] (0,4);      
\end{circuitikz}
\caption{Q.No.16}
\label{fig:1}
\end{figure}

\begin{oneparchoices}
    \choice $I_x$
    \choice $I_x/2$
    \CorrectChoice $I_x/4$
    \choice $I_x/3$
\end{oneparchoices}

\question The voltage across $660 \Omega$ resistance is (refer Figure~\ref{fig:2}):
\begin{figure}[H]
\centering
\begin{circuitikz}[american ]
\draw
(0,0) to [short, i=$I_x$] (0,4)
      to [R, l= $330 \Omega$] (4,4)
      to [R, l= $660 \Omega$] (8,4)
      to [short]  (8,0);
\draw      
(0,0) to [V, v= $3V$] (8,0);
\draw
(4,4) to [short] (4,3)
      to [R, l_= $660 \Omega$] (8,3);

\end{circuitikz}
\caption{Q.No.17}
\label{fig:2}
\end{figure}
\begin{oneparchoices}
    \choice $0.65 V$
    \choice $1.5 V$
    \choice $0.72 V$
    \CorrectChoice $0.75 V$
    
\end{oneparchoices}
\question The current $I_x$ and $I_y$ are (refer Figure~\ref{fig:3}).
\begin{figure}[H]
\centering
\begin{circuitikz}[american]
\draw
(0,0) to [isource, l=$1A$] (4,0)
      to [R, l= $5 \Omega$, i=$I_x$] (8,0)
      to [short] (8,2)
      to [short] (4,2)
      to [isource, l=$5A$] (4,0)
      to [R, l = $1 \Omega$, i=$I_y$] (4, -3)
      to [short] (0, -3)
      to [short] (0,0);
\end{circuitikz}
\caption{Q.No.18}
\label{fig:3}
\end{figure}
\begin{oneparchoices}
    \choice $-1A, 5A$
    \choice $ 5A, 1A$
    \choice $1A, 5A$
    \CorrectChoice $5A, -1A$
    
\end{oneparchoices}

\question The current $I_1$ and $I_2$ of the circuit shown in Figure~\ref{fig:4} are giving by: 
\begin{figure}[H]
\centering
\begin{circuitikz}[american]
\draw
(0,0) to [isource, l=$10A$] (0,4)
      to [short]  (8,4)
      to [R, l=$2 \Omega$, i=$I_2$] (8,0)
      to [short] (0,0) ;
\draw
(4,4) to [R, l=$3 \Omega$, i=$I_1$] (4,0);
\end{circuitikz}
\caption{Q.No.19}
\label{fig:4}
\end{figure}
\begin{oneparchoices}
    \CorrectChoice $4A, 4A$
    \choice $6A, 6A$
    \choice $4A, 6A$
    \choice $6A, 4A$
\end{oneparchoices}

\question Referring to the circuit of the Figure~\ref{fig:5}, a $35V$ source is connected to a series circuit of $600 \Omega$ and $R$. If a voltmeter of internal resistance $1.2 \SI{}{k \Omega}$ is connected across $600 \Omega$, it reads $5V$. The value of $R$ is
\begin{figure}[H]
\centering
\begin{circuitikz}[american]
\draw
(0,4) to [battery2, l=$35V$] (0,0)
      to [short] (4,0)
      to [R, l=$R$] (4,2)
      to [R, l=$600\Omega$] (4,4)
      to [short] (0,4);
\draw
(6,4) to [voltmeter] (6, 2)
      to [short] (4,2);
\draw      
(4,4) to [short] (6,4);            
\end{circuitikz}
\caption{Q.No.20}
\label{fig:5}
\end{figure}
\begin{oneparchoices}
    \CorrectChoice $1.2 \SI{}{k\Omega}$
    \choice $2.4 \SI{}{k\Omega}$
    \choice $1.4 \SI{}{k\Omega}$
    \choice $3.4 \SI{}{k\Omega}$
\end{oneparchoices}

\question  The equivalent resistance of the circuit given in Figure~\ref{fig:6} is given by
\begin{figure}[H]
\centering
\begin{circuitikz}[american, scale=0.8]
\draw
(-1,0) to [short, o-] (0,0);
\draw
(0,0) to [R, l=$1\textrm{k}\Omega$] (2,0)
      to [R, l=$1\textrm{k}\Omega$] (6,0)
      to [R, l=$3\textrm{k}\Omega$] (10,0)
      to [R, l=$5\textrm{k}\Omega$] (14,0)
      to [short] (15,0)
      to [short] (15,-4)
      to [short, -o] (-1,-4);
\draw
(3,-1) to [R, l_=$1\textrm{k}\Omega$] (5, -1);
\draw
(3, -1) to [short] (3,0);
\draw
(5, -1) to [short] (5,0);
\draw
(7,-1) to [R, l_=$3\textrm{k}\Omega$] (9, -1);
\draw
(7, -1) to [short] (7,0);
\draw
(9, -1) to [short] (9,0);
\draw
(11,-1) to [R, l_=$5\textrm{k}\Omega$] (13, -1);
\draw
(11, -1) to [short] (11,0);
\draw
(13, -1) to [short] (13,0);

\end{circuitikz}
\caption{Q.No.21}
\label{fig:6}
\end{figure}
\begin{oneparchoices}
    \choice $\SI{4}{k\Omega}$
    \CorrectChoice $\SI{10}{k\Omega}$
    \choice $\SI{5.5}{k\Omega}$
    \choice $\SI{5}{k\Omega}$
\end{oneparchoices}
\question Find current $I$, (refer Figure~ref{fig:7}).
\begin{figure}[H]
\centering
\begin{circuitikz}[american]
\draw
(0,0) to [isource, l=$1A$] (4,0)
      to [R, l=$2 \Omega$] (2,-2)
      to [R, l=$5 \Omega$] (0,0);
\draw
(0,0) to [short] (0,1.5)
      to [R, l=$10 \Omega$] (4,1.5)
      to [short] (4,0);
\draw
(2,-2) to [short] (0,-2)
       to [isource, l=$2A$] (0,0);
\draw
(2,-2) to [short] (4,-2)
       to [isource, l_=$4A$] (4,0);  
           
\end{circuitikz}
\caption{Q.No.22}
\label{fig:7}
\end{figure}
\begin{oneparchoices}
    \choice $\dfrac{17}{12}$
    \CorrectChoice $\dfrac{11}{17}$
    \choice $\dfrac{12}{17}$
    \choice $\dfrac{17}{11}$
\end{oneparchoices}
\question Number of components in VLSI are --------:\\[0.3cm]
\begin{oneparchoices}
    \choice less that $99$
    \CorrectChoice greater than $ 10,000$
    \choice $100 - 999$
    \choice $ 1,000 - 9,999 $
\end{oneparchoices}
\question A single flip flop is a modulo ------ counter.\\[0.3cm]
\begin{oneparchoices}
    \choice 0
    \CorrectChoice 1
    \choice 2
    \choice 3
\end{oneparchoices}
\question $D$ flip flop can be made from $J-K$ flip flop by making: \\[0.3cm]
\begin{oneparchoices}
    \choice $J = K$
    \CorrectChoice $J = K = 1$
    \choice $J = 0, K = 1$
    \choice $ J = \bar{K}$
\end{oneparchoices}
\vspace{0.10in}

\end{questions}
\end{document}
 